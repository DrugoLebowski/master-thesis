\documentclass{beamer}
\bibliographystyle{apalike}

\usetheme{metropolis}           % Use metropolis theme
\title{Supervised Learning of Neural Random-Access Machines with Differential Evolution}
\date{\today}
\author{Valerio Belli}
\institute{Università degli Studi di Perugia}
\titlegraphic{\hfill\includegraphics[height=1.5cm]{logo.png}}

\begin{document}
  \maketitle
  \section{Neural Random-Access Machines}
  \begin{frame}{What is?}
	It is a machine introduced in \cite{NRAM:2016} based on a neural network (MLP or LSTM) which is capable of manipulating pointers and dereferencing them. Its objective is to execute a \`{logical} circuit to solve a task on which it trained.
  \end{frame}
  \begin{frame}{Registers only version}
  	
  \end{frame}
  \begin{frame}{Memory augmented version}
  	
  \end{frame}
  \begin{frame}{Circuit example}
 		
  \end{frame}
  
  \section{Artificial Neural Network}
  \begin{frame}{Description}
  
  \end{frame}
  \begin{frame}{Perceptron}
  	
  \end{frame}
  \begin{frame}{Multi Layer Perceptron (MLP)}
  	
  \end{frame}
  
  \section{Differential Evolution}
  \begin{frame}{Description}
  
  \end{frame}
  \begin{frame}{Differential Evolution variants}
  
  \end{frame}
  \begin{frame}{Mutation variants}
  	
  \end{frame}
  \begin{frame}{Crossover variants}
  	
  \end{frame}
  \begin{frame}{DENN (Differential Evolution for Neural Network)}
  
  \end{frame}
  \section{Implementation and results}
  \begin{frame}{Implementation}
  
  \end{frame}
  \begin{frame}{Results}
  
  \end{frame}
  \begin{frame}{References}[allowframebreaks]
        \bibliography{bibliography.bib}
\end{frame}
\end{document}