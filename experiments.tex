\label{experiments}
In this chapter we present the problems with which the two solutions of NRAM have been trained, finishing by comparing the results and by presenting the circuits learned by the neural networks.

\section{Tasks}
The following are the description of the executed task used in our experiments. All except the last are the same of \cite{NRAM:2016}. In the description, big and small letters represents respectively arrays and pointers, \textit{NULL} denotes the value 0 and is used as an ending character or in the lists, as a placeholder for missing next element.

\begin{itemize}
	\item[1]{\textbf{Access} Given a value $k$ and an array \textbf{A}, return $\textbf{A}[k]$. Input is given as $k,\ A[0],\ \dots,\ $\\$\textbf{A}[n-1],\ \textit{NULL}$ and the network should replace the first memory cell with $\textbf{A}[k]$.}
	\item[2]{\textbf{Increment} Given an array $\textbf{A}$, increment all its elements by 1. Input is given as $\textbf{A}[0],\ \dots,\ \textbf{A}[n-1],\ \textit{NULL}$ and the expected output is $\textbf{A}[0] + 1,\ \dots,\ A[n-1] + 1$.}
	\item[3]{\textbf{Copy} Given an array and a pointer to the destination, copy all elements from the array to the given location. Input is given as $p,\ \textbf{A}[0],\ \dots,\ \textbf{A}[n-1]$ where $p$ points to one element after $\textbf{A}[n-1]$. The expected output is $\textbf{A}[0],\ \dots,\ \textbf{A}[n-1]$ at positions $p,\ \dots,\ p+n-1$ respectively.}
	\item[4]{\textbf{Reverse} Given an array and a pointer to the destination, copy all elements from the array in reversed order. Input is given as $p,\ \textbf{A}[0],\ \dots,\ \textbf{A}[n-1]$ where $p$ points one element after $\textbf{A}[n-1]$. The expected output is $\textbf{A}[n-1],\ \dots,\ \textbf{A}[0]$ at positions $p,\ \dots,\ p+n-1$ respectively.}
	\item[5]{\textbf{Swap} Given two pointers $p,\ q$ and an array \textbf{A}, swap elements $\textbf{A}[p]$ and $\textbf{A}[q]$. Input is given as $p,\ q,\ \textbf{A}[0],\ \dots,\ \textbf{A}[p],\ \dots,\ \textbf{A}[q],\ \dots,\ \textbf{A}[n-1],\ 0$. The expected modified array \textbf{A} is: $\textbf{A}[0],\ \dots,\ \textbf{A}[q],\ \dots,\ \textbf{A}[p],\ \dots,\ \textbf{A}[n-1]$.}
	\item[6]{\textbf{Permutation} Given two arrays of n elements: P (contains a permutation of numbers $0,\ \dots,\ n-1$) and \textbf{A} (contains random elements), permutate \textbf{A} according to P. Input is given as a, $P[0],\ \dots,\ P[n-1],\ \textbf{A}[0],\ ...,\ \textbf{A}[n-1]$, where a is a pointer to the array \textbf{A}. The expected output is $\textbf{A}[P[0]],\ \dots,\ \textbf{A}[P[n-1]]$, which should override the array P.}
	\item[6]{\textbf{Sum} Given pointers to 2 arrays \textbf{A} and \textbf{B}, and the pointer to the output $o$, sum the two arrays into one array. The input is given as: $a,\ b,\ o,\ \textbf{A}[0],\ \dots,\ \textbf{A}[n-1],\ G,\ \textbf{B}[0],\ \dots,\ \textbf{B}[m-1],\ G$, where $a$ points to first element of \textbf{A}, $b$ points to the first element of \textbf{B}, $o$ points to first element of output array and $G$ is a special guardian value. The $\textbf{A}+\textbf{B}$ array should be written starting from position $o$.}
\end{itemize}

\section{Results}

\subsection{Circuits}