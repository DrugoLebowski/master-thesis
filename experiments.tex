\label{experiments}
In this chapter we present the problems with which the two solutions of NRAM have been trained, finishing by comparing the results and by presenting the circuits learned by the neural networks.

\section{Tasks}
The following are the description of the executed task used in our experiments. All except the last are the same of \cite{NRAM:2016}. In the description, big and small letters represents respectively arrays and pointers, \textit{NULL} denotes the value 0 and is used as an ending character or in the lists, as a placeholder for missing next element.

\begin{itemize}
	\item[1]{\textbf{Access} Given a value $k$ and an array \textbf{A}, return $\textbf{A}[k]$. Input is given as $k,\ A[0],\ \dots,\ $\\$\textbf{A}[n-1],\ \textit{NULL}$ and the network should replace the first memory cell with $\textbf{A}[k]$.}
	\item[2]{\textbf{Increment} Given an array $\textbf{A}$, increment all its elements by 1. Input is given as $\textbf{A}[0],\ \dots,\ \textbf{A}[n-1],\ \textit{NULL}$ and the expected output is $\textbf{A}[0] + 1,\ \dots,\ A[n-1] + 1$.}
	\item[3]{\textbf{Copy} Given an array and a pointer to the destination, copy all elements from the array to the given location. Input is given as $p,\ \textbf{A}[0],\ \dots,\ \textbf{A}[n-1]$ where $p$ points to one element after $\textbf{A}[n-1]$. The expected output is $\textbf{A}[0],\ \dots,\ \textbf{A}[n-1]$ at positions $p,\ \dots,\ p+n-1$ respectively.}
	\item[4]{\textbf{Reverse} Given an array and a pointer to the destination, copy all elements from the array in reversed order. Input is given as $p,\ \textbf{A}[0],\ \dots,\ \textbf{A}[n-1]$ where $p$ points one element after $\textbf{A}[n-1]$. The expected output is $\textbf{A}[n-1],\ \dots,\ \textbf{A}[0]$ at positions $p,\ \dots,\ p+n-1$ respectively.}
	\item[5]{\textbf{Swap} Given two pointers $p,\ q$ and an array \textbf{A}, swap elements $\textbf{A}[p]$ and $\textbf{A}[q]$. Input is given as $p,\ q,\ \textbf{A}[0],\ \dots,\ \textbf{A}[p],\ \dots,\ \textbf{A}[q],\ \dots,\ \textbf{A}[n-1],\ 0$. The expected modified array \textbf{A} is: $\textbf{A}[0],\ \dots,\ \textbf{A}[q],\ \dots,\ \textbf{A}[p],\ \dots,\ \textbf{A}[n-1]$.}
	\item[6]{\textbf{Permutation} Given two arrays of n elements: P (contains a permutation of numbers $0,\ \dots,\ n-1$) and \textbf{A} (contains random elements), permutate \textbf{A} according to P. Input is given as a, $P[0],\ \dots,\ P[n-1],\ \textbf{A}[0],\ ...,\ \textbf{A}[n-1]$, where a is a pointer to the array \textbf{A}. The expected output is $\textbf{A}[P[0]],\ \dots,\ \textbf{A}[P[n-1]]$, which should override the array P.}
	\item[6]{\textbf{Sum} Given pointers to 2 arrays \textbf{A} and \textbf{B}, and the pointer to the output $o$, sum the two arrays into one array. The input is given as: $a,\ b,\ o,\ \textbf{A}[0],\ \dots,\ \textbf{A}[n-1],\ G,\ \textbf{B}[0],\ \dots,\ \textbf{B}[m-1],\ G$, where $a$ points to first element of \textbf{A}, $b$ points to the first element of \textbf{B}, $o$ points to first element of output array and $G$ is a special guardian value. The $\textbf{A}+\textbf{B}$ array should be written starting from position $o$.}
\end{itemize}

\section{Results}
In the experiments we have retraced what is do in the paper \cite{NRAM:2016}, trying to test the same tasks. We have do that combining the algorithms \textbf{JADE}, \textbf{SHADE} and \textbf{L-SHADE} with the mutation methods \textbf{DEGL} and \textbf{Curr to p best} and the crossover method \textbf{bin}. We show the best configurations in the results Tables \ref{tbl:tests-configurations-access}-\ref{tbl:tests-configurations-reverse} - in them are showed only the configurations that are converged. So it is clear that \textbf{L-SHADE} not have never converged. Moreover, we have used always the same configurations of NRAM, like number of registers and maximum integer, among the tests. Hence, to not overload the tables they are showed in the Section \ref{subsec:circuits} associated to the generated circuits. The cost calculation has been done with the cost function showed in the Section \ref{subsec:cost-function}, that evaluates the manipulated input memory respect to the desired memory, according to the cost mask.\newline\newline
Overall, we have found a good set of training hyperparameters only for tasks \textbf{Access}, \textbf{Copy}, \textbf{Increment} and \textbf{Reverse} - Differential Evolution has always achieved cost zero without particular efforts, differently to ADAM as can be seen in Figures \ref{fig:access-plot}-\ref{fig:reverse-plot}. Unfortunately, for the other tasks, the performed tests have not produced any good results.\newline

\begin{table}[]
	\centering
	\begin{tabular}{|c|c|c|c|c|}
		\hline
		\multicolumn{5}{|c|}{\textbf{Access}} \\ \hline
		DE type & JADE & JADE & SHADE & SHADE \\ \hline
		Mutation & DEGL & Curr to p best & DEGL & Curr to p best  \\ \hline
		Crossover & bin & bin & bin & bin \\ \hline
		Population & 60 & 60 & 60 & 60 \\ \hline
		Training gen. & 1000 & 1000 & 1000 & \\ \hline
		Convergence gen. & & & & \\ \hline
		F & & & & \\ \hline
		CR & & & & \\ \hline
		Archive size & & & & \\ \hline
		Memory \textbf{M} Size & $\times$ & $\times$ & & \\ \hline
		DEGL neighbors & 5 & $\times$ & 5 & $\times$ \\ \hline
		Hidden Layer & $2 \times 260$ & $2 \times 260$ & $2 \times 260$ &  $2 \times 260$\\ \hline
		Activation function & 2 $\times$ ReLu & 2 $\times$ ReLu & 2 $\times$ ReLu & 2 $\times$ ReLu \\ \hline
	\end{tabular}
	\caption{Best configurations used in the tests of \textbf{Access}}
	\label{tbl:tests-configurations-access}
\end{table}

\begin{table}[]
	\centering
	\begin{tabular}{|c|c|c|c|c|}
		\hline
		\multicolumn{5}{|c|}{\textbf{Increment}} \\ \hline
		DE type & JADE & JADE & SHADE & SHADE \\ \hline
		Mutation & DEGL & Curr to p best & DEGL & Curr to p best  \\ \hline
		Crossover & bin & bin & bin & bin \\ \hline
		Population & 60 & 60 & 60 & 60 \\ \hline
		Training gen. & 1000 & 1000 & 1000 & \\ \hline
		Convergence gen. & & & & \\ \hline
		F & & & & \\ \hline
		CR & & & & \\ \hline
		Archive size & & & & \\ \hline
		Memory \textbf{M} Size & $\times$ & $\times$ & & \\ \hline
		DEGL neighbors & 5 & $\times$ & 5 & $\times$ \\ \hline
		Hidden Layer & $2 \times 260$ & $2 \times 260$ & $2 \times 260$ &  $2 \times 260$\\ \hline
		Activation function & 2 $\times$ ReLu & 2 $\times$ ReLu & 2 $\times$ ReLu & 2 $\times$ ReLu \\ \hline
	\end{tabular}
	\caption{Best configurations used in the tests of \textbf{Increment}}
	\label{tbl:tests-configurations-increment}
\end{table}

\begin{table}[]
	\centering
	\begin{tabular}{|c|c|c|c|c|}
		\hline
		\multicolumn{5}{|c|}{\textbf{Copy}} \\ \hline
		DE type & JADE & JADE & SHADE & SHADE \\ \hline
		Mutation & DEGL & Curr to p best & DEGL & Curr to p best  \\ \hline
		Crossover & bin & bin & bin & bin \\ \hline
		Population & 60 & 60 & 60 & 60 \\ \hline
		Training gen. & 1000 & 1000 & 1000 & \\ \hline
		Convergence gen. & & & & \\ \hline
		F & & & & \\ \hline
		CR & & & & \\ \hline
		Archive size & & & & \\ \hline
		Memory \textbf{M} Size & $\times$ & $\times$ & & \\ \hline
		DEGL neighbors & 5 & $\times$ & 5 & $\times$ \\ \hline
		Hidden Layer & $2 \times 260$ & $2 \times 260$ & $2 \times 260$ &  $2 \times 260$\\ \hline
		Activation function & 2 $\times$ ReLu & 2 $\times$ ReLu & 2 $\times$ ReLu & 2 $\times$ ReLu \\ \hline
	\end{tabular}
	\caption{Best configurations used in the tests of \textbf{Copy}}
	\label{tbl:tests-configurations-copy}
\end{table}


\begin{table}[]
	\centering
	\begin{tabular}{|c|c|c|c|c|}
		\hline
		\multicolumn{5}{|c|}{\textbf{Reverse}} \\ \hline
		DE type & JADE & JADE & SHADE & SHADE \\ \hline
		Mutation & DEGL & Curr to p best & DEGL & Curr to p best  \\ \hline
		Crossover & bin & bin & bin & bin \\ \hline
		Population & 60 & 60 & 60 & 60 \\ \hline
		Training gen. & 1000 & 1000 & 1000 & \\ \hline
		Convergence gen. & & & & \\ \hline
		F & & & & \\ \hline
		CR & & & & \\ \hline
		Archive size & & & & \\ \hline
		Memory \textbf{M} Size & $\times$ & $\times$ & & \\ \hline
		DEGL neighbors & 5 & $\times$ & 5 & $\times$ \\ \hline
		Hidden Layer & $2 \times 260$ & $2 \times 260$ & $2 \times 260$ &  $2 \times 260$\\ \hline
		Activation function & 2 $\times$ ReLu & 2 $\times$ ReLu & 2 $\times$ ReLu & 2 $\times$ ReLu \\ \hline
	\end{tabular}
	\caption{Best configurations used in the tests of \textbf{Reverse}}
	\label{tbl:tests-configurations-reverse}
\end{table}

\begin{table}[]
\centering
\label{my-label} 
\begin{tabular}{|c|c|c|c|}
\hline
\textbf{Task} & \textbf{Train complexity} & \textbf{Train error} & \textbf{Generalization} \\ \hline

Access & $\textrm{len}(A) \leq 20$ & 0 & Perfect \\ \hline
Increment & $\textrm{len}(A) \leq 10$ & 0 & Perfect \\ \hline
Copy & $\textrm{len}(A) \leq 6$ & 0 & Perfect \\ \hline
Reverse & $\textrm{len}(A) \leq 6$ & 0 & Perfect \\ \hline

\end{tabular}
\caption{Configurations used in tests}
\label{tbl:tests-configuration}
\end{table}

All the found parameters generalize perfectly with sequences much longer respect to those used in the training. The generalization tests have been performed with sequences of maximum 100 values. For the other problems, including \textbf{Swap} which in the original paper is considered as a ``Simple'' task, the experiments have not produced any good results. During the training both solutions never reached cost zero, indicating that the controller does not learns the algorithm. The error of each task, associated to the complexity, is visible in the Figure \ref{fig:error-hard-tasks}. As a hypothesis, we impute this to the incapacity of the Differential Evolution to find a controller which uses all the pointers in the memories.

\subsection{Circuits}\label{subsec:circuits}